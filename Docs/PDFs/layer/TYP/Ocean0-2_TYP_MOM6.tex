% !TEX TS-program = pdflatex
% !TEX encoding = UTF-8 Unicode

% This is a simple template for a LaTeX document using the "article" class.
% See "book", "report", "letter" for other types of document.

\documentclass[11pt]{article} % use larger type; default would be 10pt

\usepackage[utf8]{inputenc} % set input encoding (not needed with XeLaTeX)

%%% Examples of Article customizations
% These packages are optional, depending whether you want the features they provide.
% See the LaTeX Companion or other references for full information.

%%% PAGE DIMENSIONS
\usepackage{geometry} % to change the page dimensions
\geometry{a4paper} % or letterpaper (US) or a5paper or....
% \geometry{margin=2in} % for example, change the margins to 2 inches all round
% \geometry{landscape} % set up the page for landscape
%   read geometry.pdf for detailed page layout information

\usepackage{graphicx} % support the \includegraphics command and options

% \usepackage[parfill]{parskip} % Activate to begin paragraphs with an empty line rather than an indent

%%% PACKAGES
\usepackage{booktabs} % for much better looking tables
\usepackage{array} % for better arrays (eg matrices) in maths
\usepackage{paralist} % very flexible & customisable lists (eg. enumerate/itemize, etc.)
\usepackage{verbatim} % adds environment for commenting out blocks of text & for better verbatim
\usepackage{subfig} % make it possible to include more than one captioned figure/table in a single float
% These packages are all incorporated in the memoir class to one degree or another...
\usepackage{natbib}

%%% HEADERS & FOOTERS
\usepackage{fancyhdr} % This should be set AFTER setting up the page geometry
\pagestyle{fancy} % options: empty , plain , fancy
\renewcommand{\headrulewidth}{0pt} % customise the layout...
\lhead{}\chead{}\rhead{}
\lfoot{}\cfoot{\thepage}\rfoot{}

%%% SECTION TITLE APPEARANCE
\usepackage{sectsty}
\allsectionsfont{\sffamily\mdseries\upshape} % (See the fntguide.pdf for font help)
% (This matches ConTeXt defaults)

%%% ToC (table of contents) APPEARANCE
\usepackage[nottoc,notlof,notlot]{tocbibind} % Put the bibliography in the ToC
\usepackage[titles,subfigure]{tocloft} % Alter the style of the Table of Contents
\renewcommand{\cftsecfont}{\rmfamily\mdseries\upshape}
\renewcommand{\cftsecpagefont}{\rmfamily\mdseries\upshape} % No bold!

\usepackage{fullpage}

%%% END Article customizations

%%% The "real" document content comes below...

\title{Summary of Ocean0-2\_TYP\_MOM6 Results}
\author{Gustavo Marques, Alon Stern, Matthew Harrison, Olga Sergienko, \\ Alistair Adcroft and Robert Hallberg}

\begin{document}
\maketitle

\section{Model Details}

\begin{itemize}
\item Model and version: Modular Ocean Model v. 6 (MOM6).
\item Repository: https://github.com/gustavo-marques/MOM6/releases/tag/ISOMIP.v1.0 
\item Model configuration and input files: 
\begin{itemize}
   \item https://github.com/gustavo-marques/ISOMIP/tree/master/setup/Ocean0/TYP
   \item https://github.com/gustavo-marques/ISOMIP/tree/master/setup/Ocean1/TYP
   \item https://github.com/gustavo-marques/ISOMIP/tree/master/setup/Ocean2/TYP
\end{itemize}

\item Vertical coordinate: layered isopycnal (density) coordinate with a bulk mixed layer scheme.

\item Horizontal mixing: harmonic (del2); along-isopycnal for diffusivities and along-layer for viscosity.

\item Vertical mixing: del2 with COM constant viscosity and diffusivity set as background values. Additional vertical mixing is also applied based on the parameterization developed by \cite{Jackson2008}, more details below. Within the mixed layer, vertical mixing is also controlled by the bulk mixed layer scheme described in \cite{Hallberg2003}.

\item Advection schemes: momentum - second-order centered; tracers - piecewise linear method.

\item Equation of state: linear with ISOMIP+ coefficients.

\item Convection parameterization: based on the parameterization developed by \cite{Jackson2008} using a critical Richardson number of $Ri_c$ = 0.25.

\item Bulk mixed layer: minimum thickness of 2 m and the minimum vertical viscosity of $10^{-2}$ m$^2$/s.

\item Melt parameterization: $T_w$, $S_w$ and $u_w$ were averaged within the mixed layer thickness computed by the bulk mixed layer scheme. $u_w$ was averaged the to the tracer grid using four horizontal neighbors. Melting was set to zero in regions where the ice depth was less than 90 m and when the total water column thickness was less than 10 m.

\item Modifications to Topography: Interpolated to 2-km grid using a spline method\footnote{http://docs.scipy.org/doc/scipy/reference/generated/scipy.interpolate.interp2d.html\#scipy.interpolate.interp2d} then smoothed the ice thickness using a Gaussian filter\footnote{http://docs.scipy.org/doc/scipy-0.16.1/reference/generated/scipy.ndimage.filters.gaussian\_filter.html} with half-width of 1 and 5 km for the y and x direction, respectively. An offline calving criterion was used where ice thinner than 100 m was removed. To minimize pressure gradient errors due to a step-like ice cliff, this criterion was not applied near the ice front, which remained smooth. A minimum thickness of $\sim$ 40 m was maintained by decreasing the ice thickness near the grounding line.  

\item Maintaining sea level: mass fluxes were used and no corrections were applied to maintain the sea level unchanged.  

\item TYP parameters: horizontal resolution of 2 km and 72 vertical layers. Following \cite{Holland1999}, $\Gamma_T$ and $\Gamma_S$ were computed based on the stability of the boundary layer. The model included frazil ice formation, where seawater was frozen if the temperature was below the freezing point.

\item TYP problem: we would typically apply this configuration in idealized (e.g.,\cite{Goldberg2012a,Goldberg2012b}) and realistic (e.g., regional and global configurations) studies. 

\end{itemize}


\bibliographystyle{te}


\bibliography{../../bib/refs}

\end{document}
