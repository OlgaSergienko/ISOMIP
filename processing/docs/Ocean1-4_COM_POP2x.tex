% !TEX TS-program = pdflatex
% !TEX encoding = UTF-8 Unicode

% This is a simple template for a LaTeX document using the "article" class.
% See "book", "report", "letter" for other types of document.

\documentclass[11pt]{article} % use larger type; default would be 10pt

\usepackage[utf8]{inputenc} % set input encoding (not needed with XeLaTeX)

%%% Examples of Article customizations
% These packages are optional, depending whether you want the features they provide.
% See the LaTeX Companion or other references for full information.

%%% PAGE DIMENSIONS
\usepackage{geometry} % to change the page dimensions
\geometry{a4paper} % or letterpaper (US) or a5paper or....
% \geometry{margin=2in} % for example, change the margins to 2 inches all round
% \geometry{landscape} % set up the page for landscape
%   read geometry.pdf for detailed page layout information

\usepackage{graphicx} % support the \includegraphics command and options

% \usepackage[parfill]{parskip} % Activate to begin paragraphs with an empty line rather than an indent

%%% PACKAGES
\usepackage{booktabs} % for much better looking tables
\usepackage{array} % for better arrays (eg matrices) in maths
\usepackage{paralist} % very flexible & customisable lists (eg. enumerate/itemize, etc.)
\usepackage{verbatim} % adds environment for commenting out blocks of text & for better verbatim
\usepackage{subfig} % make it possible to include more than one captioned figure/table in a single float
% These packages are all incorporated in the memoir class to one degree or another...

%%% HEADERS & FOOTERS
\usepackage{fancyhdr} % This should be set AFTER setting up the page geometry
\pagestyle{fancy} % options: empty , plain , fancy
\renewcommand{\headrulewidth}{0pt} % customise the layout...
\lhead{}\chead{}\rhead{}
\lfoot{}\cfoot{\thepage}\rfoot{}

%%% SECTION TITLE APPEARANCE
\usepackage{sectsty}
\allsectionsfont{\sffamily\mdseries\upshape} % (See the fntguide.pdf for font help)
% (This matches ConTeXt defaults)

%%% ToC (table of contents) APPEARANCE
\usepackage[nottoc,notlof,notlot]{tocbibind} % Put the bibliography in the ToC
\usepackage[titles,subfigure]{tocloft} % Alter the style of the Table of Contents
\renewcommand{\cftsecfont}{\rmfamily\mdseries\upshape}
\renewcommand{\cftsecpagefont}{\rmfamily\mdseries\upshape} % No bold!

\usepackage{fullpage}

%%% END Article customizations

%%% The "real" document content comes below...

\title{Summary of Ocean1--4\_COM\_POP2x Results}
\author{Xylar Asay-Davis}

\begin{document}
\maketitle

\section{Model Details}

\begin{itemize}
\item Model and version: Parallel Ocean Program v. 2x (POP2x)
\item Repository: not publicly available.
\item Vertical coordinate: $z$~level with partial top and bottom cells.
\item Horizontal mixing: harmonic (del2) along geopotentials.
\item Vertical mixing: del2 with COM constant viscosity and diffusivity.
\item Advection schemes: momentum: centered, tracers: flux limited Lax-Wendroff.
\item Equation of state: linear with ISOMIP+ coefficients.
\item Convection parameterization: enhanced vertical mixing (ISOMIP+ values of $\nu_\textrm{unstab}$ and 
$\kappa_\textrm{unstab}$)
\item Melt parameterization: $T_w$ and $S_w$ are computed by averaging $T$ and $S$ with 20\,m of the ice
draft.  $u_w$ is averaged over 4 ``horizontal`` neighbors (at the ice--ocean interface) from the velocity to the tracer 
grid but is not averaged vertically.
\item Modifications to Topography: Interpolated to 2-km grid with conservative interpolation scheme, 
smoothed with a Gaussian filter with half-width of 2\,km.  A minimum thickness of 2 grid cells (40\,m) was 
maintained by deepening bathymetry near the grounding line.  Partial top cells thinner than 5\,m are either 
thickened or removed.  Ice draft and bathymetry are automatically adjusted to ensure required connectivity 
between neighboring cells (e.g.removing or horizontally expanding cells with no horizontal neighbors).
\item Maintaining sea level: Using virtual salt fluxes, so sea-level change is negligible. 
\item Moving boundaries (Ocean3--4): Topography data was interpolated to monthly snapshots, calving criterion
was applied.  After updating topography, T and S were progressively extrapolated first horizontally, then 
vertically into new ocean cells.  Barotropic transport was maintained by re-distributing barotropic velocity across 
the expanded or contracted water column.  Velocities are initially zero in new ocean columns.  T, S and v are simply 
zeroed in cells removed from the ocean.
\item Deviations from COM: none.
\item Parameter values:

\begin{tabular}{rl}
$\Gamma_T$ & $0.1146$ \\
$\Gamma_S$ & $3.27429 \times 10^{-3}$ \\
$C_{D,\textrm{top}}$ & $2.5 \times 10^{-3}$
\end{tabular}
\end{itemize}

\end{document}
